%! TEX root = ../axions.tex
\begin{equation*}
    % \feynmandiagram [small,inline=(t1.base) horizontal=t2 to p1] {
    % t1[crossed dot] -- t2 -- t3 -- t1,
    % t2 -- [photon] p1 [particle=\(\gamma\)],
    % t3 -- [photon] p2 [particle=\(\gamma\)],
    % p1 -- [opacity=0] p2,
    % };
    \begin{tikzpicture}
        \begin{feynman}
            \diagram [small, vertical=t3 to t2] {
            t1[crossed dot] -- t2 -- t3 -- t1,
            t2 -- [photon] p2 [particle=\(\gamma\)],
            t3 -- [photon] p1 [particle=\(\gamma\)],
            p1 -- [opacity=0] p2,
            };
            \end{feynman}
    \end{tikzpicture}
    \begin{tikzpicture}
        \begin{feynman}
            \diagram [small, vertical=t3 to t2] {
            t1[crossed dot] -- t2 -- t3 -- t1,
            t2 -- [draw=none] p2 [particle=\(\gamma\)],
            t3 -- [draw=none] p1 [particle=\(\gamma\)],
            p1 -- [opacity=0] p2,
            };
            \diagram*{
                (t2)--[photon](p1),
                (t3)--[photon](p2),
            };
            \end{feynman}
    \end{tikzpicture}  
\end{equation*}
