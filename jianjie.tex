%! TEX root = ./axions.tex
\section{简介}
首先,什么是轴子?
\begin{enumerate}
   \item 轴子场是周期性场,即$a(x)=a(x)+2\pi f_a$,其中$f_a$是轴子的衰变常数。
   \item 轴子场和pseudo shift-symmetry相关联。
   \item 轴子场是$U(1)_{PQ}$对称性自发破缺产生的pseudo Goldstone Boson。
   \item 轴子场是用来解决强CP问题(强相互作用中CP不破坏)而被提出。
\end{enumerate}
\subsection{EFT以及QCD反常}
考虑一个复场$\phi$,具有$U(1)$对称性$\phi:\stackrel{U(1)}{\longmapsto} e^{i\alpha}\phi$,假设$Q=1$。则拉氏量的写法为
\begin{equation}
   \mL = \partial_{\mu} \phi^{*} \partial^{\mu} \phi-V(\phi^*\phi),\ V=\lambda(\phi^*\phi-\frac{f^2}{2})^2.
\end{equation}
令$x=\phi^*\phi$,则
\begin{equation*}
   0=\frac{\partial V}{\partial x}=2\lambda(x-\frac{f^2}{2})\Rightarrow \langle \phi^*\phi\rangle = \frac{f^2}{2}.
\end{equation*}
%! TEX root = ../axions.tex
\begin{figure}[h]
    \centering
    \begin{tikzpicture}
        [x={(0.8cm,0cm)},z={(0cm,0.1cm)},y={(-0.4cm,-0.7cm)}]
        \draw[->,thick,black!70] (0,0,0) -- (4,0,0) node[right] {Im$\phi$}; % 虚轴
        \draw[->,thick,black!70] (0,0,0) -- (0,4,0) node[below right] {Re$\phi$}; % 实轴
        \draw[->,thick,black!70,scale=0.1] (0,0,0) -- (0,0,600) node[above] {V}; % 势能 
        
        % ellipse
            \def \rx{1.5}
            \def \ry{1}
            \def \z{0}
            \draw[rotate around={-12:(0,0,0)},dashed]  ellipse({\rx} and {\ry});
            \def \nx{2.88}
            \def \ny{1.44}
            \def \tz{41}
            \def \tx{-0.6}
            \draw[rotate around={-7:(0,0,0)},dashed] (\tx,0,\tz
            ) ellipse({\nx} and {\ny});
        % function
        \draw[domain=-3:3,samples=200,smooth]
        plot(\x,0,{(\x*\x-2.5)^2});

        \end{tikzpicture}
\end{figure}

重新参数化$\phi$为$\phi=\frac{f+\rho}{\sqrt{2}}e^{ia/f}$。此时$U(1)$变换为
\begin{align}
   \begin{aligned}
      \phi = \frac{f+\rho}{\sqrt{2}}e^{ia/f}&\rightarrow e^{i\alpha}\phi\\
      a &\rightarrow a + \alpha f.
   \end{aligned}
\end{align}
场$a$的对称性变为shift-symmetry。

将拉氏量用新的参数描述
\begin{equation}
   \mL = \frac{1}{2}\partial_{\mu}\rho \partial^{\mu}\rho + \frac{1}{2}(1+\frac{\rho}{f})^2\partial_{\mu}a\partial^{\mu}a-\frac{1}{2}m_{\rho}^2\rho^2-\frac{m_{\rho}^2}{2f}\rho^3-\frac{m_{\rho}^2}{8f^2}\rho^4,
\end{equation}
其中$m_{\rho}^2=2\lambda f^2$。
\textcolor{blue}{$E<<f,f\ge 10^3GeV$}。


\textbf{ChPT简介}

手征微扰场论是QCD的低能有效理论。考虑QCD的拉氏量
\begin{equation}
   \mL_{QCD} = \sum_{i=1}^{3} (\bar{q}_L i \sD q_L + \bar{q}_R i \sD q_R  )-(m_{ij} \bar{q}_{Ri}q_{Lj}+{\rm h.c.})-\frac{1}{4}GG+\theta \frac{g_s^2}{32\pi^2}G\tilde{G},
\end{equation}
其中$G,\tilde{G}$省略了指标。

在$m_{q_i}\to 0$的极限下,系统具有$U(3)_L\times U(3)_R$的整体对称性,向下分解为$SU(3)_L\times SU(3)_R\times U(1)_L\times U(1)_R)$。其中与$SU(3)_L\times SU(3)_R$关联的诺特流为
\begin{equation*}
   j_{L,R}^{\mu,a}=\bar{q}_{L,R}\gamma^{\mu}T^a q_{L,R},\ T^a=\frac{\lambda_a}{2},
\end{equation*}
与$U(1)_L\times U(1)_R$关联的诺特流为
\begin{equation*}
   j_{L,R}^{\mu}=\bar{q}_{L,R}\gamma^{\mu}q_{L,R}.
\end{equation*}

采用另一组基$V=L+R,A=R-L$,我们可以重新将对称性改写为$SU(3)_V\times SU(3)_A\times U(1)_V\times U(1)_A$,诺特流改写为
\begin{align}
   \begin{aligned}
      j^{\mu,a}=\bar{q}\gamma^{\mu}T^aq,&\ j_5^{\mu,a}=\bar{q}\gamma^{\mu}\gamma_5 T^a q\\
      j^{\mu}=\bar{q}\gamma^{\mu}q,& \ j_5^{\mu}=\bar{q}\gamma^{\mu}\gamma_5 q.
   \end{aligned}
\end{align}
然而质量项在$SU(3)_L\times SU(3)_R$下不是不变的。为此我们将质量项提升为场,并令其在$SU(3)_L\times SU(3)_R$下按照如下方式变换
\begin{equation*}
   M_q \to LM_qR^{\dagger}.
\end{equation*}
在最后的时候,我们再令
\begin{equation*}
   M_q \to \begin{pmatrix}
      m_u&&\\
      &m_d&\\
      &&m_s.
   \end{pmatrix}
\end{equation*}

除此显式破缺之外,QCD的真空在$SU(3)_L\times SU(3)_R$下也是自发破缺的。
\begin{equation*}
   \langle 0 |\bar{q}_{R,j}q_{L,i}|0\rangle=\hat{\Lambda}^3\delta_{ij}, 
\end{equation*}
其中$[\Lambda]$带有质量量纲。上式在$SU(3)_L\times SU(3)_R$变换下
\begin{equation*}
   \to \langle 0 |\bar{q}_{R,l}q_{L,k}|0\rangle R^{\dagger}_{lj} L_{ik}=\hat{\Lambda}^3 (LR^{\dagger})_{ij}=\hat{\Lambda}^3\Sigma_{ij}.
\end{equation*}
$\Sigma_{ij}$代表了不同于$\delta_{ij}$的真空。如果$L=R$,那么真空是不变的,对应$SU(3)_V$对称性。因此我们有$SU(3)_L\times SU(3)_R\to SU(3)_V$,破缺的部分生成了8个Goldstone Bosons。

分析$U(1)$对称性的破缺,我们可以发现,参数化后的场$a$(即Goldstone Boson)连接了不同的真空。因此对于$SU(3)$,我们同样可以将8个Goldstone Bosons参数化为连接不同真空的变换,即将$\Sigma_{ij}$提升为Goldstone Boson场。
\begin{equation*}
   \begin{aligned}
   &\Sigma = e^{i\frac{2\pi^a T^a}{f_{\pi}}}, T^a=\frac{\lambda_a}{2}\\
   &\Sigma\stackrel{SU(3)_L\times SU(3)_R}{\longrightarrow}L\Sigma R^{\dagger}\\
   &\pi^a T^a = \begin{pmatrix}
      \pi^0+\frac{\eta}{3} & \sqrt{2}\pi^+ & \sqrt{2}K^+ \\
      \sqrt{2}\pi^- & -\pi^0+\frac{\eta}{3} &\sqrt{2}K^0 \\
      \sqrt{2}K^- & \sqrt{2}\bar{K}^0 & -\frac{2\eta}{3}
   \end{pmatrix}.
   \end{aligned}
\end{equation*}
忽略重的粒子态(如$\rho,\eta$),我们可以将EFT按照$\frac{m_q}{\Lambda}$的幂级数进行展开,其中$\Lambda\sim 1GeV$。

写下$\mO(p^2)$阶的拉氏量
\begin{equation}
   \mL_2 = \frac{f_{\pi}^2}{4}{\rm Tr}[\partial_{\mu}\Sigma^{\dagger}\partial^{\mu}\Sigma]+\frac{B_0 f_{\pi}^2}{2}{\rm Tr}[\Sigma M_q^[\dagger]+M_q\Sigma^{\dagger}].
\end{equation}
通过将上式展开,我们可以得到
\begin{align}
   \begin{aligned}
      m_{\pi^{\pm}}^2 = B_0(m_u+m_d)\\
      m_{K^{\pm}}^2 = B_0(m_u+m_s)\\
      m_{\bar{K}^0}^2 = B_0(m_d+m_s)
   \end{aligned}
\end{align}

\textbf{低能QCD中的反常}

考虑无质量的QED项
\begin{align}
   \mL = i\bar{\Psi}(\spa-ie\sA)\Psi - \frac{1}{4}F_{\mu\nu}F^{\mu\nu}.
\end{align}
相应的整体对称性以及Noether流有
\begin{align}
   \begin{cases}
      \Psi \to e^{i\alpha}\Psi\\
      \Psi \to e^{i\gamma_5\alpha}\Psi
   \end{cases},
   \begin{cases}
      j^{\mu}_V=\bar{\Psi}\gamma^{\mu}\Psi\\
      j^{\mu}_A=\bar{\Psi}\gamma^{\mu}\gamma_5\Psi.
   \end{cases}
\end{align}
在经典情况下,两个流都是守恒的。然而在量子水平上,无法保证两个流同时守恒。如果我们选择$\partial_{\mu}j^{\mu}_V=0$,那么就有
\begin{align}
   \partial_{\mu}j^{\mu}_A = \frac{e^2}{32\pi^2}\varepsilon^{\mu\nu\rho\sigma}F_{\mu\nu}F_{\rho\sigma}=\frac{e^2}{16\pi^2}F\tilde{F}.
\end{align}
可以通过下面的费曼图计算得到

%! TEX root = ../axions.tex
\begin{equation*}
    % \feynmandiagram [small,inline=(t1.base) horizontal=t2 to p1] {
    % t1[crossed dot] -- t2 -- t3 -- t1,
    % t2 -- [photon] p1 [particle=\(\gamma\)],
    % t3 -- [photon] p2 [particle=\(\gamma\)],
    % p1 -- [opacity=0] p2,
    % };
    \begin{tikzpicture}
        \begin{feynman}
            \diagram [small, vertical=t3 to t2] {
            t1[crossed dot] -- t2 -- t3 -- t1,
            t2 -- [photon] p2 [particle=\(\gamma\)],
            t3 -- [photon] p1 [particle=\(\gamma\)],
            p1 -- [opacity=0] p2,
            };
            \end{feynman}
    \end{tikzpicture}
    \begin{tikzpicture}
        \begin{feynman}
            \diagram [small, vertical=t3 to t2] {
            t1[crossed dot] -- t2 -- t3 -- t1,
            t2 -- [draw=none] p2 [particle=\(\gamma\)],
            t3 -- [draw=none] p1 [particle=\(\gamma\)],
            p1 -- [opacity=0] p2,
            };
            \diagram*{
                (t2)--[photon](p1),
                (t3)--[photon](p2),
            };
            \end{feynman}
    \end{tikzpicture}  
\end{equation*}


整体流和局部流的反常:

整体:会导致重要的物理后果,\textcolor{blue}{如,QCD中的$U(1)_A$反常能够推出$m_q\sim 1GeV$;$\pi^0\to\gamma\gamma$现象,QCD中的$U(1)_{PQ}$反常能够推出轴子的质量。}

局部:\textcolor{blue}{会破坏理论的可重整性。}

反常来自于路径积分中测度的变换:
\begin{align}
   \mD q \mD \bar{q} \stackrel{q\to e^{i\alpha \gamma_5 q}}{\longrightarrow} \mD q\mD \bar{q} e^{-i\int \md^4x[\frac{e^2}{16\pi^2F\tilde{F}}]}.
\end{align}
一个整体流反常有如下公式:
\begin{align}
   \partial_{\mu}j^{\mu,a}=\frac{g^2}{32\pi^2}{\rm Tr}[T^a\{t^b,t^c\}]\varepsilon^{\mu\nu\rho\sigma}F^b_{\mu\nu}F^c_{\rho\sigma}.
\end{align}

考虑$SU(3)$群,我们有轴矢流$j^{\mu}_5=\bar{q}\gamma^{\mu}\gamma_5 \mathbb{I} q$,其中$q^T=(u,d,s)$,因此
\begin{align}
   \begin{aligned}
      \partial_{\mu}j^{\mu}_5 =&\frac{g_s^2}{32\pi^2}{\rm Tr}[{\mathbb{I}}\{t^b,t^c\}]\varepsilon G^bG^c\\
      =&\frac{g_s^2}{32\pi^2}{\rm Tr}[{\mathbb{I}}]\cdot2{\rm Tr}[t^bt^c]\varepsilon G^bG^c\\
      =&\frac{g_s^2N_f}{32\pi^2}\varepsilon G^bG^b \to \textcolor{blue}{m_{\eta^{\prime}}\sim 1GeV.}
   \end{aligned}
\end{align}
其中的${\mathbb{I}}$是因为该诺特流与味空间无关。在最后一步,我们用了${\rm Tr}{\mathbb{I}}=N_f$,以及${\rm Tr}[t^bt^c]=\frac{1}{2}\delta^{bc}$。

考虑$U(1)_{EM}$群,我们有轴矢流$j_5^{\gamma,a}=\bar{q}\gamma^{\mu}\gamma_5T^aq$,以$a=3$为例,其中$Q={\rm diag}\{2/3,-1/3,-1/3\}$,
\begin{align}
   \begin{aligned}
      \partial_{\mu}j^{\mu,3}_5=&\frac{e^2}{32\pi^2}{\rm Tr}[T^3\{Q,Q\}]\varepsilon FF\\
      =&\frac{e^2}{32\pi^2}(\frac{N_C}{3})\varepsilon FF.
   \end{aligned}
\end{align}
\textcolor{blue}{在最后一步,我们用了$2{\rm Tr}[T^3 Q^2]=(\frac{N_C}{3})$}。
因此,积分测度变为
\begin{align}
   \mD q\mD\bar{q}\stackrel{q\to e^{i\alpha T^3\gamma_5}q}{\longrightarrow} \mD q\mD\bar{q}e^{-i\int \md^4x\alpha \frac{e^2}{32\pi^2}(\frac{N_C}{3})\varepsilon FF} ,
\end{align}
拉氏量变为
\begin{align}
   \mL \to \mL -\alpha \frac{e^2}{32\pi^2}(\frac{N_C}{3})\varepsilon FF.
\end{align}

考虑带有$\pi$介子的理论,假设系统沿着$T^3$方向进行$SU(3)_A$变换,可以证明$\pi^0\to\pi^0-\alpha f_{\pi}$。因此,为了保证$\mL_{\pi}$和$\mL_{QCD}$的变换方式一样,我们需要添加一项
\begin{align}
   \delta \mL_{\pi^0}=\frac{\pi^0}{f_{\pi}}\frac{e^2}{32\pi^2}\frac{N_C}{3}\varepsilon FF,
\end{align}
该项通常称为反常匹配,该项在$SU(3)_A$变化下为
\begin{align}
   \delta mL_{\pi^0}\to\delta mL_{\pi^0}\-\frac{\alpha f_{\pi}}{f_{\pi}}\frac{e^2}{32\pi^2}\frac{N_C}{3}\varepsilon FF,
\end{align}
刚好抵消了拉氏量的变化。而额外引入的$\delta\mL_{\pi^0}$项则会产生如下顶点:

%! TEX root = ../axions.tex
\begin{center}
    \feynmandiagram [medium, horizontal=a to b] {
    a[particle=$\pi^0$] --[scalar] b,
    b --[photon] t1[particle=$\gamma$],
    b --[photon] t2[particle=$\gamma$],
    };
\end{center}


由此,我们可以给出
\begin{align}
   \Gamma(\pi^0\to \gamma\gamma)=\frac{m_{\pi}^3\alpha^3}{64\pi^3f_{\pi}^2}(\frac{N_C}{3})\simeq 7.6eV (\frac{N_C}{3}),
\end{align}
而$\Gamma(\pi^0\to \gamma\gamma)_{\rm exp}\simeq 7.7eV$。

